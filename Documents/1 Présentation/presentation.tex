\documentclass[12pt,a4paper]{article}
\usepackage[utf8]{inputenc}
\usepackage[french]{babel}
\usepackage[T1]{fontenc}
\usepackage{hyperref}
\author{Tien-Thinh}
\title{Présentation TIPE}
\date{2021-2022}


\begin{document}
\maketitle
\section*{Exploration \& Recherche d'environnement 3D}

\section{Introduction}
Le TIPE 2021-2022 est sur "Santé, Prévention". \\
Etant donné le nombre de personne à vacciner, j'ai voulu créer un programme qui sache vaccinner automatiquement une personne. (Pour la faisabilité, nous testerons le vaccinage sur un ballon de baudruche sans le faire éclater).

\section{Grandes Lignes}
\begin{itemize}
\item Nuages de Points 3D à partir d'images
\item Reconnaissance d'objet à étudier 
\item Controller un bras articulé
\end{itemize} 

\section{Etat de l'art}
\subsection{Nuages de points 3D}
Les nuages de points sont un ensemble de points dans l'espace.
\subsubsection{Conversion en Surface 3D}
\url{https://en.wikipedia.org/wiki/Point_cloud}
Pour faire des surfaces : 
\begin{itemize}
\item \href{https://en.wikipedia.org/wiki/Polygon_mesh}{Polygon mesh} : \href{https://en.wikipedia.org/wiki/Delaunay_triangulation}{Delaunay Triangulation} ou \href{https://en.wikipedia.org/wiki/Alpha_shape}{alpha-shape}
\item \href{https://en.wikipedia.org/wiki/Triangle_mesh}{Triangle mesh}
\item \href{https://en.wikipedia.org/wiki/Non-uniform_rational_B-spline}{NBS Mesh}
\item CAD model
\end{itemize}




\end{document}